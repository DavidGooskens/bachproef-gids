\chapter{Voorbereiding: een werkomgeving opzetten}
\label{ch:voorbereiding}

In dit hoofdstuk behandelen we het opstarten van het werk aan een bachelorproef. Je vindt er enkele aanbevelingen over te gebruiken tools en het onderzoeksproces.

\section{Versiebeheersysteem}
\label{sec:versiebeheersysteem}

% Gebruik een versiebeheersysteem zoals Git
% -> Github of Bitbucket om remote een

\section{Gebruik van {\LaTeX}}
\label{sec:gebruik-van-latex}

De meeste studenten zijn gewend om opgemaakte tekst met een klassieke tekstverwerker (typisch MS Word) te schrijven. Voor je bachelorproef is het aangewezen om hier van af te stappen.

Word, zeker met het standaardsjabloon, geeft een layout die niet geschikt is voor publicatie. Eens de lengte en complexiteit van een Word-document toenemen (en bij een eindwerk is dat zeker het geval), krijg je te maken met inconsistenties in de layout van je tekst, paginanummering, slecht gepositioneerde afbeeldingen, enz.

Wanneer je tekst kopieert vanuit een ander (voorbereidend) document, wordt de oorspronkelijke layout overgenomen. Als die niet consistent is met deze van je hoofddocument, moet je alles gaan aanpassen.

Een klassieke tekstverwerker die gebaseerd is op het WYSIWYG-principe\footnote{\emph{What You See Is What You Get}, zoals je wellicht weet}, laat je toe om tot in de puntjes te bepalen waar tekst op het papier terecht komt, maar in dit geval is dat te veel vrijheid. Een strakke en professionele vormgeving is een specialiteit die een grote aandacht voor vaak pietluttige  details vraagt. Als informaticus hebben wij niet de nodige kennis om dit te realiseren. Wanneer je een significant deel van de tijd bezig bent met het vormgeven van je document, word je bovendien afgeleid van de kern van de zaak: de inhoud van de tekst!

% Word kan niet in versiebeheer => verschillende versies vh document naast elkaar
% ("bachproef 3" "bachproef 5 30 maart" "final draft" "final draft na feedback" "final final draft", ...)

% LaTeX is een tekstzetsysteem met markuptaal (vgl HTML) die gespecialiseerd is in het op papier zetten van tekst met een professionele layout. Schrijf broncode in LaTeX markup, compiler genereert PDF.
% LaTeX is tekstgebaseerd, dus je kan dit in een versiebeheersysteem steken (zie vorige sectie).

\section{Bibliografische databank}
\label{sec:bibliografische-databank}

\section{Samenvatting}
\label{sec:voorbereiding-samenvatting}

De kernpunten van dit hoofdstuk zijn:

\begin{itemize}
  \item Gebruik een versiebeheersysteem om al je werk in op te slaan (Git is aanbevolen);
  \item Schrijf je tekst in {\LaTeX} in plaats van een klassieke tekstverwerker voor een strakke, professionele opmaak;
  \item Gebruik een \emph{reference manager} voor het bijhouden van een bibliografische databank (JabRef of Mendeley zijn aanbevolen).
\end{itemize}
