\chapter{Literatuuronderzoek}
\label{ch:literatuuronderzoek}

% Belang literatuurstudie
% - zelf vertrouwd worden met het onderwerp
% - lezer context onderwerp geven, alle info die nodig is om het onderwerp te begrijpen
% - in academische thesis: *alles* dat er over het onderwerp te vinden is
% - 

Alle beweringen in een thesis (dus ihb ook in een bachelorproef) moeten onderbouwd zijn, hetzij aan de hand van referenties naar gezaghebbende literaire bronnen, hetzij aan de hand van eigen ontwikkelde kennis, op basis van methodologisch correct verzamelde en geanalyseerde data.

%- Cursus informatievaardigheden HoGent bib
%- bibliografische databank
    %- BibTeX tips: {} voor familienaam en bedrijfsnaam; online bronnen; ...
%- bronnen zoeken
%- kwaliteit van bronnen evalueren
  %- criteria
  %- bronnen die zeker niet in aanmerking komen: Wikipedia (of andere encyclopedie), webpagina van onbekende auteur/publicatiedatum
%- bronnen gebruiken en er naar verwijzen, toepassing in Mendeley en JabRef
  %- Wetenschappelijk artikel
  %- Vaktijdschrift
  %- Boek
  %- Thesis (doctoraat, master, bachelor)
  %- Hoofdstuk in boek (bv. elk hst andere auteur)
  %- Blog-artikel
  %- Presentatie op conferentie, video
  %- Handleiding software
  %- White paper
  %- Webpagina
%- literatuurstudie schrijven
  %- verwijzingen in de tekst
  %- literatuurlijst

% De eerste stap in het uitwerken van een bachelorproef is de literatuurstudie. Ik wil jullie graag herinneren aan enkele interessante informatiebronnen daarover:
%
% De HoGent bib heeft een {http://bib.hogent.be/cursus/overzicht/}{cursus informatievaardigheden} en een sectie over {http://bib.hogent.be/hoekanik/scripties/citeren/}{citeren en refereren}
% Neem opnieuw de cursus Onderzoekstechnieken bij de hand, in het bijzonder het lesmateriaal ivm LaTeX en rapporteren over onderzoek.
% Vergeet niet om al wat je vindt op te slaan in een bibliografische databank, bijvoorbeeld JabRef of Mendeley. Met zorg de nodige metadata over de gevonden bronnen opslaan neemt behoorlijk wat tijd in beslag, maar dit is de moeite waard.
%
% Goede startpunten voor het zoeken naar informatie:
%
% Via de HoGent bib krijg je toegang tot een grote hoeveelheid wetenschappelijke en vakliteratuur die niet publiek beschikbaar zijn.
% Zoek naar scripties, thesissen, bachelorproeven van vorige jaren over je onderwerp. http://bib.hogent.be/zoeken/scripties-hogent/
% Log in op Apollo (https://apollo.hogent.be/, via VPN) voor toegang tot databanken die niet publiek beschikbaar zijn. In het bijzonder Elsevier ScienceDirect, Springer Online Journals, en Web of Science zijn voor ons vakgebied het interessantst.
% Vanuit Apollo kan je ook Google Scholar lanceren. Dat kan ook via het publieke internet, maar dan heb je geen toegang tot bepaalde bronnen. Scholar kent de abonnementen waar de bib op is ingeschreven en kan op die manier toegang geven tot artikels die niet publiek toegankelijk zijn.
% Springer eBooks http://bib.hogent.be/zoeken/ebooks/ is een verzameling van boeken uitgegeven door Springer, met o.a. een uitgebreid aanbod binnen computerwetenschappen.
% Via het publieke internet vind je uiteraard ook veel informatie.
% Wikipedia is een goed startpunt, maar vergeet niet dat Wikipedia-artikels op zich niet kunnen als referenties. Bekijk de oorspronkelijke bronnen van het artikel.
% Arxiv.org is een database van Open Acces artikels in een hele reeks onderzoeksdomeinen, o.a. de Computing Research Repository http://arxiv.org/corr/home
% Ga op zoek naar presentaties op vakconferenties rond je onderwerp (vb. Google IO, WWDC, http://lanyrd.com/topics/android/, \ldots). Tegenwoordig worden vele conferenties gefilmd en achteraf gepubliceerd op Youtube of Vimeo
% Er bestaan verschillende portaalsites voor actuele ict-gerelateerde onderwerpen waar technische artikels, presentaties, interviews, enz. op verschijnen, bv. dzone.com, infoq.com, enz.
% Wie zijn de belangrijkste namen in de ``community''? Keynote-speakers op conferenties, auteurs van de belangrijkste boeken over het onderwerp, enz. Volg deze personen op Twitter, zoek uit of ze een blog hebben, actief zijn op LinkedIn, enz. Lees al wat je kan vinden dat ze geschreven hebben de laatste jaren.
