\chapter{Literatuuronderzoek}
\label{ch:literatuuronderzoek}

De eerste fase in elk onderzoek is typisch om een overzicht te schrijven van de huidige stand van zaken in het onderzoeksdomein. Daarvoor is het nodig om je te verdiepen in al wat er over het onderwerp al geschreven is. Dit is het literatuuronderzoek. In elk verslag over onderzoek is het essentieel dat je elke bewering die je doet ook kan aantonen. Dat kan ofwel op basis van data die je zelf op een methodologisch correcte manier verzameld en geanalyseerd hebt (maar dat komt verder in deze gids aan bod), ofwel aan de hand van refererenties naar \emph{gezaghebbende} publicaties.

In dit hoofdstuk gaan we dieper in op dit onderwerp: wat bereik je precies met een literatuurstudie, hoe kan je er aan beginnen, hoe kan je de bronnen die je vindt gestructureerd bijhouden en op welke manier gebruik je die dan in je tekst.

\section{Doel van het literatuuronderzoek}
\label{sec:doel-literatuuronderzoek}

Het belangrijkste doel van een literatuuronderzoek is om vertrouwd te worden met het onderzoeksdomein. Je gaat dus zoveel mogelijk informatie verzamelen (en lezen) over het onderwerp zodat je er eigenlijk alles over weet dat er op dit moment over te weten valt. Het is ook de bedoeling om alle kennis die je op die manier hebt opgedaan ook op een gestructureerde manier samen te vatten. Dit wordt meestal het eerste hoofdstuk van je bachelorproef.

Aan de hand van de literatuurstudie geef je de lezer de nodige achtergrond om het onderwerp te begrijpen. Het is een inleiding op het onderwerp, en bespreekt de huidige stand van zaken. Je vermeldt wat de experten in het domein er over te zeggen hebben en welk onderzoek er in het verleden al over gedaan is (met uiteraard vermelding van de belangrijkste conclusies). Uit de literatuurstudie moet ook naar voor komen dat er nog hiaten in onze kennis zijn, dat er een probleem is dat om een oplossing vraagt. En dat is uiteraard precies het onderwerp van je bachelorproef.

\section{Soorten bronnen}
\label{sec:soorten-bronnen}

In het kader van (toegepast) onderzoek kunnen we informatiebronnen ondeverdelen in deze drie categorieën:

\begin{description}
  \item[Primaire] kennis die je zelf vergaart tijdens je onderzoek. Bijvoorbeeld metingen uit experimenten, resultaten van enquêtes, transcripties van interviews, enz.
  \item[Secundaire] publicatie van kennis, onderzoek, enz.~door anderen. Bijvoorbeeld artikels in vaktijdschriften, boek, presentatie op een conferentie, enz.
  \item[Tertiaire] zoekindexen en encyclopedieën. Bijvoorbeeld Google Scholar, Wikipedia, Web of Science, Elsevier ScienceDirect, Arxiv.org, enz.
\end{description}

Wanneer je in een tekst verwijst naar de literatuur, dan gaat het telkens over \emph{secundaire} bronnen. Dat betekent dat tertiaire bronnen \emph{niet} kunnen. Je mag dus bijvoorbeeld niet verwijzen naar een Wikipedia-artikel.

%- Cursus informatievaardigheden HoGent bib
%- bibliografische databank
    %- BibTeX tips: {} voor familienaam en bedrijfsnaam; online bronnen; ...
%- bronnen zoeken
%- kwaliteit van bronnen evalueren
  %- criteria
  %- bronnen die zeker niet in aanmerking komen: Wikipedia (of andere encyclopedie), webpagina van onbekende auteur/publicatiedatum
%- bronnen gebruiken en er naar verwijzen, toepassing in Mendeley en JabRef
  %- Wetenschappelijk artikel
  %- Vaktijdschrift
  %- Boek
  %- Thesis (doctoraat, master, bachelor)
  %- Hoofdstuk in boek (bv. elk hst andere auteur)
  %- Blog-artikel
  %- Presentatie op conferentie, video
  %- Handleiding software
  %- White paper
  %- Webpagina
%- literatuurstudie schrijven
  %- verwijzingen in de tekst
  %- literatuurlijst

% De eerste stap in het uitwerken van een bachelorproef is de literatuurstudie. Ik wil jullie graag herinneren aan enkele interessante informatiebronnen daarover:
%
% De HoGent bib heeft een {http://bib.hogent.be/cursus/overzicht/}{cursus informatievaardigheden} en een sectie over {http://bib.hogent.be/hoekanik/scripties/citeren/}{citeren en refereren}
% Neem opnieuw de cursus Onderzoekstechnieken bij de hand, in het bijzonder het lesmateriaal ivm LaTeX en rapporteren over onderzoek.
% Vergeet niet om al wat je vindt op te slaan in een bibliografische databank, bijvoorbeeld JabRef of Mendeley. Met zorg de nodige metadata over de gevonden bronnen opslaan neemt behoorlijk wat tijd in beslag, maar dit is de moeite waard.
%
% Goede startpunten voor het zoeken naar informatie:
%
% Via de HoGent bib krijg je toegang tot een grote hoeveelheid wetenschappelijke en vakliteratuur die niet publiek beschikbaar zijn.
% Zoek naar scripties, thesissen, bachelorproeven van vorige jaren over je onderwerp. http://bib.hogent.be/zoeken/scripties-hogent/
% Log in op Apollo (https://apollo.hogent.be/, via VPN) voor toegang tot databanken die niet publiek beschikbaar zijn. In het bijzonder Elsevier ScienceDirect, Springer Online Journals, en Web of Science zijn voor ons vakgebied het interessantst.
% Vanuit Apollo kan je ook Google Scholar lanceren. Dat kan ook via het publieke internet, maar dan heb je geen toegang tot bepaalde bronnen. Scholar kent de abonnementen waar de bib op is ingeschreven en kan op die manier toegang geven tot artikels die niet publiek toegankelijk zijn.
% Springer eBooks http://bib.hogent.be/zoeken/ebooks/ is een verzameling van boeken uitgegeven door Springer, met o.a. een uitgebreid aanbod binnen computerwetenschappen.
% Via het publieke internet vind je uiteraard ook veel informatie.
% Wikipedia is een goed startpunt, maar vergeet niet dat Wikipedia-artikels op zich niet kunnen als referenties. Bekijk de oorspronkelijke bronnen van het artikel.
% Arxiv.org is een database van Open Acces artikels in een hele reeks onderzoeksdomeinen, o.a. de Computing Research Repository http://arxiv.org/corr/home
% Ga op zoek naar presentaties op vakconferenties rond je onderwerp (vb. Google IO, WWDC, http://lanyrd.com/topics/android/, \ldots). Tegenwoordig worden vele conferenties gefilmd en achteraf gepubliceerd op Youtube of Vimeo
% Er bestaan verschillende portaalsites voor actuele ict-gerelateerde onderwerpen waar technische artikels, presentaties, interviews, enz. op verschijnen, bv. dzone.com, infoq.com, enz.
% Wie zijn de belangrijkste namen in de ``community''? Keynote-speakers op conferenties, auteurs van de belangrijkste boeken over het onderwerp, enz. Volg deze personen op Twitter, zoek uit of ze een blog hebben, actief zijn op LinkedIn, enz. Lees al wat je kan vinden dat ze geschreven hebben de laatste jaren.
