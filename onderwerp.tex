\chapter{Een onderwerp kiezen}
\label{ch:onderwerp}

% Onderzoeksdomein
% Onderzoeksvraag + deelvragen
% Titel formuleren
%  - geen afkortingen, vakjargon
%  - lang genoeg, concreet
% Hoe vind je een onderwerp?
%
% ga op zoek naar de actualiteit in het domein dat je het meeste interesseert:
%  - Er bestaan verschillende portaalsites voor actuele ict-gerelateerde onderwerpen waar technische artikels, presentaties, interviews, enz. op verschijnen, bv. dzone.com, infoq.com, enz.
%  - Ga op zoek naar relevante conferenties (vb. Google IO, WWDC, http://lanyrd.com/topics/android/, \ldots). Vaak vind je video-opnamen van de presentaties op Youtube/Vimeo/\ldots
%  - Wie zijn de belangrijkste namen in de ``community''? Keynote-speakers op conferenties, auteurs van de belangrijkste boeken over het onderwerp, enz.
%  - Volg deze personen op Twitter, zoek uit of ze een blog hebben, actief zijn op LinkedIn, enz. Lees al wat je kan vinden dat ze geschreven hebben de laatste jaren.
%  - Newsletters, bv. Cron.Weekly, Devops Weekly
%
%  Dan zou je in principe de belangrijkste thema's moeten herkennen waar men op dit moment vooral mee bezig is en dat zou inspiratie kunnen geven voor je onderwerp.
%

