\chapter{Een onderwerp kiezen}
\label{ch:onderwerp}

In dit hoofdstuk worden enkele suggesties gegeven voor het zoeken naar een onderwerp. De website van de HoGent bib heeft daar ook enkele algemene raadgevingen over\footnote{\url{https://bib.hogent.be/how-to/onderwerp-formuleren/inleiding/}}, deze gids is specifiek gericht op de bachelor toegepaste informatica.

Vanuit de opleiding krijgen we van externen regelmatig aanbiedingen van onderwerpen die geschikt zijn voor een bachelorproef, maar dat aanbod is niet groot genoeg om alle studenten van een onderwerp te voorzien. Langs de andere kant is het zelf uitwerken van een onderwerp een interessante kans om je te verdiepen in een onderwerp waar je na je afstuderen graag mee zou verder gaan.

\section{Het onderzoeksdomein kiezen}
\label{sec:het_onderzoeksdomein_kiezen}

Een eerste stap is het kiezen van je onderzoeksdomein. Dit is iets waar eigenlijk niemand je mee kan helpen. Kies een domein waar je zelf in geïnteresseerd bent, zodat je voldoende motivatie kan opbrengen om je je hier in te verdiepen. Je gekozen specialisatierichting in het laatste jaar is wellicht een goed startpunt. In welk soort job zou je na je afstuderen graag starten? Met welke technologieën/platformen/\ldots zou je het liefst werken?

Een eerste stap is om op zoek te gaan naar de actualiteit binnen je gekozen vakdomein. Enkele tips:

\begin{itemize}
  \item Er bestaan verschillende portaalsites voor actuele ict-gerelateerde onderwerpen waar technische artikels, presentaties, interviews, enz. op verschijnen, bv. dzone.com\footnote{\url{https://dzone.com/}}, infoq.com\footnote{\url{https://www.infoq.com/}}, enz.
  \item Ga op zoek naar voor je vakgebied relevante conferenties, workshops, symposia, enz. Je hoeft je daarbij niet te beperken tot conferenties in eigen land! Voorbeelden zijn Devoxx\footnote{\url{https://devoxx.be/}} (Java), Google IO\footnote{\url{https://events.google.com/io2016/}} (Android), WWDC\footnote{\url{https://developer.apple.com/wwdc/}} (iOS), Configuration Management Camp\footnote{\url{http://cfgmgmtcamp.eu/}} (Linux Systeembeheer), enz. Lanyrd\footnote{\url{http://lanyrd.com/topics/}} is een website waar je vele conferenties kan terugvinden over allerlei onderwerpen en verspreid over heel de wereld.

    Meer en meer worden de lezingen van conferenties gefilmd en gepubliceerd op Youtube of Vimeo \footnote{Voor Devoxx bijvoorbeeld op \url{https://www.youtube.com/user/parleysdotcom}}. Je kan ook de sprekers opzoeken en nagaan of ze hun slides gepubliceerd hebben op Slideshare\footnote{\url{https://slideshare.net/}} of Speakerdeck\footnote{\url{https://speakerdeck.com/}}.
  \item Zijn er lokaal verenigingen die geïnteresseerd zijn in je vakgebied? Bijvoorbeeld OWASP Belgium\footnote{\url{https://www.owasp.org/index.php/Belgium}} (beveiliging van mobiele en webapplicaties). Je kan naar zulke groepen zoeken via Meetup\footnote{\url{https://meetup.com/}} of ook via LinkedIn\footnote{\url{https://www.linkedin.com/}} (zoek specifiek naar groepen zoals het Belgian IT Infrastructure Network\footnote{\url{https://www.linkedin.com/groups/2092569}}). Kijk eens na of er binnenkort evenementen in de buurt zijn en ga er naartoe.
  \item Wie zijn de belangrijkste namen in de ``community''? Keynote-speakers op conferenties, auteurs van de belangrijkste boeken over het onderwerp, enz. Volg deze personen op Twitter, zoek uit of ze een blog hebben, actief zijn op LinkedIn, enz. Lees al wat je kan vinden dat ze geschreven hebben de laatste jaren.
  \item Zoek uit of er nieuwsbrieven bestaan over je onderwerp, die periodiek updates uit sturen over de actualiteit binnen dat vakgebied. Voorbeelden zijn Cron.Weekly\footnote{\url{https://www.cronweekly.com/}} (Linux systeembeheer) of DevOps Weekly\footnote{\url{http://www.devopsweekly.com/}}.
\end{itemize}

Belangrijk is ook je tijd te nemen om je ``onder te dompelen'' in wat er binnen je gekozen vakgebied aan het gebeuren is. Dit lukt niet op een avond. Het is efficiënter om hier gedurende een aantal weken regelmatig wat tijd in te steken (bv. elke dag een uur). Op de duur zou je in principe de belangrijkste thema's moeten herkennen waar men op dit moment vooral mee bezig is en dat zou inspiratie kunnen geven voor je onderwerp.

\section{Onderzoeksvraag formuleren}
\label{sec:onderzoeksvraag_formuleren}

Eens je voeling krijgt met de actualiteit van een onderwerp, leer je typisch ook de belangrijkste problemen en discussiepunten kennen. Die kunnen aanleiding geven tot het formuleren van je hoofdonderzoeksvraag, die je verder kan opsplitsen in concretere deelonderzoeksvragen.

% TODO: goed voorbeeld geven.
% Zo concreet mogelijk, meetbaar maken
% SMART 

Op goede onderzoeksvraag bestaat er nu nog geen sluitend antwoord. Vragen als ``Wat is data mining?'' of ``Welke PHP frameworks zijn er?'' zijn dus niet geschikt, want het antwoord is snel te vinden door even te zoeken op Wikipedia of Google.

\section{Onderwerp uitschrijven}
\label{sec:onderwerp_uitschrijven}

Eens je een onderzoeksvraag hebt, kan je je onderwerp uitschrijven om het in te dienen ter goedkeuring. Dat betekent dat je ook al wat literatuuronderzoek gaat uitvoeren. Voor aanwijzingen over de aanpak hiervan, zie Hoofdstuk~\ref{ch:literatuuronderzoek}.

Wat je leest over het onderwerp, moet je dan structureren en formuleren in je eigen woorden in een doorlopende tekst. Een goed hulpmiddel om je gedachten over een onderwerp te ordenen om er later een gestructureerde tekst rond te schrijven is het opzetten van een mindmap. Er bestaan hiervoor verschillende tools die je kosteloos kan gebruiken, zoals bijvoorbeeld XMind\footnote{\url{https://www.xmind.net/}} of FreeMind\footnote{\url{http://freemind.sourceforge.net/}}.

% TODO: voorbeeld mindmap?

In dit stadium is het nog niet de bedoeling een volledig uitgewerkte literatuurstudie uit te schrijven. Hou je bij het minimum dat nodig is om de lezer te laten begrijpen wat de context van je onderzoek is en waarom er een probleem is dat om een oplossing vraagt. Dit is wél het startpunt van je literatuurstudie.

Denk ook na over een titel voor je bachelorproef. Die hoeft nog niet definitief te zijn, maar een goede titel maakt duidelijk welke richting je precies wil uitgaan. Een concrete titel geeft je begeleider(s) het vertrouwen dat je weet wat je precies wil gaan doen en dat dit een realistische doelstelling is. Probeer te letten op het volgende bij het formuleren van een titel:

\begin{itemize}
  \item Formuleer de titel niet als een vraag.
  \item Gebruik geen vakjargon en zeker geen afkortingen. De titel moet ook begrijpbaar zijn voor iemand buiten jouw specifieke vakgebied.
  \item Enkel je vakdomein benoemen is onvoldoende want veel te vaag. Je titel moet concreet zijn en duidelijk maken wat je precies wil onderzoeken. Dat betekent dus ook dat een titel gerust lang mag zijn. bv. ``Cloud computing'' is een algemene term die veel ladingen dekt, en die dus niets zegt. ``De selectie van een open source Infrastructure as a Service platform voor het opzetten van een testomgeving voor webontwikkeling'' is een stuk concreter.
\end{itemize}

% Niet speculeren over de toekomst.

% Waaraan moet een onderwerp voldoen
% * Er is een concrete, duidelijk afgebakende onderzoeksvraag, onderzoeksdoelstelling
% * Voorstel is vernieuwend en heeft een duidelijke meerwaarde voor een specifieke doelgroep uit het ict-werkveld
% * De methodologie is duidelijk verantwoord, onderzoekstechnieken zijn geschikt voor beantwoorden onderzoeksvraag
% * Er is een duidelijke eigen bijdrage en technische diepgang

% Zorg dat je niet te veel afhankelijk bent van externe factoren. Als je bv. beweert dat je interviews gaat uitvoeren bij bedrijven, zorg dat je op voorhand al de nodige contacten hebt. -> Als het niet lukt om tijdig de nodige personen te kunnen spreken, is het resultaat van je BP in gevaar.
