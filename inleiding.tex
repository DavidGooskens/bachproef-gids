\chapter{Inleiding}
\label{sec:inleiding}


% TODO
% Een scriptie is een afgesloten geheel. Alle informatie die nodig is om het onderwerp ten gronde te begrijpen moet er in zitten. Het mag dus niet nodig zijn om nog externe bronnen na te lezen voordat je de tekst kan begrijpen.
%
% Doelpubliek. Over het algemeen moet de tekst begrijpbaar zijn voor een informaticus die niet noodzakelijk vertrouwd is met het onderwerp. Je kan dus al een zekere basiskennis veronderstellen, maar termen en afkortingen specifiek voor je onderzoeksdomein moeten uitgelegd worden.
%
% Onderwerp ten gronde uitspitten
%
% Geen vage kwalificaties: lang, groot, snel, populair, \ldots Quantificeer je uitspraken met cijfers en meeteenheden!
%
% Alle beweringen in een thesis (dus ihb ook in een bachelorproef) moeten onderbouwd zijn, hetzij aan de hand van referenties naar gezaghebbende literaire bronnen, hetzij aan de hand van eigen ontwikkelde kennis, op basis van methodologisch correct verzamelde en geanalyseerde data.
%
% Volgorde van werken
% - werkomgeving opzetten
% - onderwerp kiezen
% - literatuurstudie
% - verzamelen en analyseren van data
% - conclusie, samenvatting en voorwoord schrijven
